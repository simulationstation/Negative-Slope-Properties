\documentclass[aps,prd,reprint,superscriptaddress,nofootinbib,floatfix]{revtex4-2}

\usepackage{amsmath,amssymb}
\usepackage{graphicx}
\usepackage{booktabs}
\usepackage{hyperref}
\usepackage{lineno}
\usepackage{url}

\begin{document}
\title{Testing the Bekenstein-Hawking Area Law at Cosmological Scales: A Data-Driven Reconstruction of the Horizon Entropy Current}

\author{Aiden B. Smith}
\affiliation{Independent Researcher}

\begin{abstract}
The Bekenstein-Hawking area law implies a constant horizon entropy slope, $\mu(A)\propto dS/dA=\mathrm{const}$. We test this assumption directly at cosmological scales using a data-driven reconstruction of the apparent-horizon entropy current from late-time background geometry (Pantheon+, BAO, and cosmic chronometers), in a framework designed to avoid GR-anchored circularity. In the baseline mapping, the reconstructed slope statistic is negative, with posterior mean $\langle d\log\mu/d\log A\rangle=-0.492$ and posterior weight $P(d\log\mu/d\log A<0)=0.904$. Across mapping variants, all converged runs keep a negative mean slope. However, decision-grade calibration under the Bekenstein-Hawking null ($\mu=1$) does not yet support a smoking-gun rejection: with 2000 null simulations, the calibrated exceedance is $p_{\rm one}=0.296$ (95\% Monte Carlo CI $[0.276,0.316]$), and low-false-alarm thresholds tuned to $\alpha=0.05$ and $0.01$ have weak power for plausible injected slopes. The current evidence is therefore a suggestive preference for negative slope, not a calibrated exclusion of the strict area law.
\end{abstract}

\maketitle
\linenumbers
\modulolinenumbers[5]

\section{Introduction}
The thermodynamic interpretation of gravity links horizon geometry to entropy. In standard General Relativity with Bekenstein-Hawking entropy, the horizon entropy is proportional to area, $S\propto A$, so $dS/dA$ is constant. A central question for late-time cosmology is whether this area law remains valid on the apparent horizon traced by the observed expansion history.

This paper is framed as a direct test of that assumption. Rather than starting from a fixed gravitational field equation and fitting parameters, we reconstruct the entropy-current proxy $\mu(A)$ from observed background geometry and then ask whether $\mu(A)=1$ is supported. In this sense, the analysis is not only a reconstruction exercise; it is a fundamental consistency test of space-time thermodynamics at cosmological scales.

The same thermodynamic degree of freedom is also the ingredient that controls modified gravitational-wave propagation in dark-siren sectors. The present work therefore serves as the theory-facing companion to dark-siren propagation tests: here we infer the background thermodynamic current, while dark sirens test its propagation consequence.

\section{Thermodynamic Link and Inference Strategy}
We define the entropy-slope modification
\begin{equation}
\mu(A)\equiv \frac{(dS/dA)_{\rm BH}}{dS/dA},
\end{equation}
so that $\mu=1$ corresponds to strict Bekenstein-Hawking behavior. In the apparent-horizon mapping used in this pipeline, the late-time background obeys
\begin{equation}
\frac{dH^2}{dz}=3H_0^2\Omega_{m0}(1+z)^2\mu(A),\qquad
A(z)=4\pi\left(\frac{c}{H(z)}\right)^2
\end{equation}
for the flat mapping variant.

The inference is performed by forward modeling: we sample a spline representation of $\log\mu$ and solve for $H(z)$, then evaluate the likelihood in data space. The baseline geometry stack is Pantheon+, BAO, and cosmic chronometers. Inference products are stress-tested with synthetic closure, simulation-based calibration (SBC), and explicit Bekenstein-Hawking null exceedance experiments using the same production settings as the real-data run.

\section{Results}
\subsection{Area-law test from reconstructed entropy current}
The reconstructed posterior favors a negative entropy-current slope over a constant-slope area law. A representative summary is
\begin{equation}
\left\langle\frac{d\log\mu}{d\log A}\right\rangle_{\rm V1,free} = -0.492,\qquad
P\!\left(\frac{d\log\mu}{d\log A}<0\right)_{\rm V1,free}=0.904.
\end{equation}
Interpreted physically, this corresponds to a decreasing $\mu(A)$ as the horizon area grows, i.e., an evolving entropy density on cosmological horizons. Sign stability across mapping variants is preserved in this run: V0\_fixedOm ($-0.483$, $P_{<0}=0.903$), V1\_curved ($-0.453$, $P_{<0}=0.881$), V1\_free ($-0.492$, $P_{<0}=0.904$), and V2\_residual ($-0.285$, $P_{<0}=0.764$).

\subsection{Decision-grade calibration and BH-null exceedance}
We calibrate thresholds directly under the Bekenstein-Hawking null using full end-to-end simulations. For each target false-alarm level $\alpha$, a threshold is selected from 2000 null simulations and then evaluated for injected-slope power (500 simulations per injected level).

At $\alpha=0.05$, the calibrated threshold is $-1.4475$, with achieved $\mathrm{FPR}=0.050$ (95\% CI $[0.041,0.060]$), but power is low: $\mathrm{TPR}=0.028,0.032,0.052$ for injected slopes $-0.3,-0.2,-0.1$ respectively. At $\alpha=0.01$, the threshold is $-1.9846$, with achieved $\mathrm{FPR}=0.010$ (95\% CI $[0.0065,0.0154]$) and $\mathrm{TPR}=0.006,0.012,0.010$ for the same injections.

Using the observed real-data scalar statistic $s_{\rm obs}=-0.4922$ (baseline V1\_free), the calibrated null exceedance from 2000 Bekenstein-Hawking simulations is
\begin{equation}
p_{\rm one}=P(s_{\rm sim}\le s_{\rm obs})=0.296,\qquad
p_{\rm two}=0.5465.
\end{equation}
With the pre-defined stop rule, these results classify the current run as \textit{suggestive preference} rather than a calibrated rejection of $\mu=1$.

\subsection{Hubble-scale anchors and robustness context}
For transfer-map stress tests, we track the conventional anchor scales $H_0^{\rm Planck}=67.4\pm0.5$ and $H_0^{\rm local}=73.0\pm1.0$ km\,s$^{-1}$\,Mpc$^{-1}$ as external references. These anchors are used for sensitivity mapping, not as hard priors in the core area-law test, to preserve modified-gravity neutrality in the reconstruction stage.

\subsection{Directional null test (Pantheon+ hemispheres and Planck lensing)}
Because the entropy-current reconstruction is built from low-redshift geometry, it is important to check that the recovered trend is not dominated by a particular sky region or survey footprint. We therefore perform an explicit statistical-isotropy diagnostic using a hemisphere-split scan on the Pantheon+ sky: for each axis on a coarse HEALPix grid, we compute the slope-scar statistic $s$ separately in opposite hemispheres and form $\Delta s$ and its uncertainty.

To suppress look-elsewhere effects, we use a five-fold crossfit procedure in which the ``best'' axis is chosen on a training split (maximizing $|\Delta s|/\sigma_{\Delta s}$) and then evaluated only once on held-out data. The held-out confirmations are consistent with noise, with fold-level $z_{\rm test}\in\{-0.354,-0.254,-0.028,+0.488,+0.065\}$ and a convenient aggregate diagnostic $z_{\rm Stouffer}\approx -0.037$.

As an external cross-check, we evaluate a hemispherical Planck 2018 CMB lensing convergence variance statistic along the strongest training axis, finding $\Delta\log{\rm Var}(\kappa)\approx 0.032$ with a random-axis calibrated fixed-axis $p$-value $p\approx 0.22$ ($z\approx 1.23$). Within current sensitivity, these diagnostics provide no evidence for a robust dipolar modulation of the inferred slope statistic.

\section{Discussion}
The key physical output is two-part: the posterior trend is directionally negative, but the calibrated decision layer does not yet reject the strict area law at low false-alarm settings. A decreasing $\mu(A)$ remains a plausible interpretation of the reconstructed geometry, yet the BH-null exceedance rate and low power under stringent FPR control imply that the present dataset and pipeline settings do not constitute smoking-gun evidence.

The calibration and closure layers are included specifically to separate posterior preference from decision-grade rejection. In this run, mapping variants agree on slope sign and converge, but the low-FPR power and null-tail probability fail the pre-specified decision threshold.

\section{Conclusion}
We recast late-time entropy-slope reconstruction as a fundamental cosmological test of the Bekenstein-Hawking area law. In the reconstructed posterior, the horizon entropy current is biased toward a negative slope and this sign preference is stable across tested mapping variants.

At the same time, decision-grade calibration with BH-null simulations shows that the present evidence is \textit{suggestive} rather than decisive: calibrated null exceedance is not small and low-FPR operation has weak power at plausible injected effect sizes. The immediate implication is methodological: stronger data, more informative cross-probe predictive tests, or higher-power test statistics are required before claiming a smoking-gun violation of $\mu(A)=\mathrm{const}$.

\section*{Data and software availability}
Software archive for this pipeline (Zenodo title: \textit{Negative Entropy Slope}): \href{https://doi.org/10.5281/zenodo.18604922}{doi:10.5281/zenodo.18604922}.

External data/product DOIs used by the underlying stack include:
\begin{itemize}
\item GWTC-3 products: \href{https://doi.org/10.1103/PhysRevX.13.041039}{doi:10.1103/PhysRevX.13.041039}.
\item GLADE+: \href{https://doi.org/10.1093/mnras/stac1443}{doi:10.1093/mnras/stac1443}.
\item Pantheon+ papers and dataset: \href{https://doi.org/10.3847/1538-4357/ac8b7a}{doi:10.3847/1538-4357/ac8b7a}, \href{https://doi.org/10.3847/1538-4357/ac8e04}{doi:10.3847/1538-4357/ac8e04}, \href{https://doi.org/10.5281/zenodo.16365279}{doi:10.5281/zenodo.16365279}.
\item Cosmic-chronometer compilations: \href{https://doi.org/10.1088/1475-7516/2012/08/006}{doi:10.1088/1475-7516/2012/08/006}, \href{https://doi.org/10.1088/1475-7516/2016/05/014}{doi:10.1088/1475-7516/2016/05/014}.
\item BOSS DR12 BAO+FS: \href{https://doi.org/10.1093/mnras/stx721}{doi:10.1093/mnras/stx721}.
\item eBOSS DR16 LRG BAO+RSD: \href{https://doi.org/10.1093/mnras/staa2455}{doi:10.1093/mnras/staa2455}.
\item DESI 2024 BAO analyses: \href{https://doi.org/10.1088/1475-7516/2025/04/012}{doi:10.1088/1475-7516/2025/04/012}, \href{https://doi.org/10.1088/1475-7516/2025/02/021}{doi:10.1088/1475-7516/2025/02/021}.
\item Planck 2018 lensing: \href{https://doi.org/10.1051/0004-6361/201833886}{doi:10.1051/0004-6361/201833886}.
\end{itemize}

\section*{AI-use statement}
AI systems were used for drafting support and editing during manuscript preparation.

\begin{thebibliography}{9}
\bibitem{gwtc3}
R.~Abbott \textit{et al.} (LIGO Scientific Collaboration, Virgo Collaboration, and KAGRA Collaboration),
Phys. Rev. X \textbf{13}, 041039 (2023),
\href{https://doi.org/10.1103/PhysRevX.13.041039}{doi:10.1103/PhysRevX.13.041039}.

\bibitem{glade}
G.~D\'alya \textit{et al.},
Mon. Not. R. Astron. Soc. \textbf{514}, 1403 (2022),
\href{https://doi.org/10.1093/mnras/stac1443}{doi:10.1093/mnras/stac1443}.
\end{thebibliography}

\end{document}

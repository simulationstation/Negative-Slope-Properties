\documentclass[aps,prd,reprint,superscriptaddress,nofootinbib,floatfix]{revtex4-2}

\usepackage{amsmath,amssymb}
\usepackage{graphicx}
\usepackage{booktabs}
\usepackage{hyperref}
\usepackage{lineno}

\begin{document}
\title{Decision-Grade Calibration of Negative-Slope Discriminability in a Dark-Siren-Motivated Growth-Mapping Pipeline}

\author{Aiden B. Smith}
\affiliation{Independent Researcher}

\begin{abstract}
This Letter reports a decision-grade calibration test of whether a dark-siren-motivated growth-mapping pipeline can distinguish relative versus absolute coupling constructions in synthetic truth-labeled injections. A fixed threshold ($\tau=0.0646$) from prior calibration is held fixed and evaluated on a large power-map suite with 192 tasks (4 truth points, 48 holdouts each, 256 replicates per holdout). For the baseline decision run, the mean true-positive rate across relative-truth points is 0.4043, the mean false-positive rate on the absolute-truth point is 0.2983, and the separation is $\Delta_{\rm sep}=0.1059$. Three additional full-core replicate seeds yield $\Delta_{\rm sep}=\{0.1000,0.1067,0.1089\}$, giving a four-run mean 0.1054 with run-to-run standard deviation 0.0038. These results show stable, nonzero out-of-sample discriminability under fixed scoring and threshold rules, but only at moderate strength. The test therefore supports a calibrated and reproducible directional signal within this model family, while remaining insufficient for model-independent physical confirmation on its own.
\end{abstract}

\maketitle
\linenumbers
\modulolinenumbers[5]

\section{Test definition}
For each replicate, let
\begin{equation}
\Delta s \equiv s_{\rm rel} - s_{\rm abs},
\end{equation}
where $s_{\rm rel}$ and $s_{\rm abs}$ are paired scores under relative and absolute growth mappings. A replicate is classified as ``relative'' if $\Delta s \ge \tau$ and ``absolute'' otherwise, using fixed threshold
\begin{equation}
\tau = 0.06459924567842923.
\end{equation}

\section{Decision-grade run}
The decision configuration used:
\begin{itemize}
\item truth grid: relative $\nu=\{0.2,0.5,0.8\}$ and absolute $\nu=0.0$;
\item holdouts per truth point: 48;
\item replicates per holdout: 256;
\item workers: 256 CPU processes;
\item run output: \texttt{outputs/growth\_mapping\_power\_map\_decision\_20260210\_044918UTC/}.
\end{itemize}

Point-level rates from the baseline decision run:
\begin{align}
\mathrm{TPR}_{\nu=0.2} &= 0.3432,\\
\mathrm{TPR}_{\nu=0.5} &= 0.3805,\\
\mathrm{TPR}_{\nu=0.8} &= 0.4892,\\
\mathrm{FPR}_{\nu=0.0} &= 0.2983.
\end{align}
Global means are
\begin{align}
\langle \mathrm{TPR}_{\rm rel} \rangle &= 0.4043,\\
\langle \mathrm{FPR}_{\rm abs} \rangle &= 0.2983,\\
\Delta_{\rm sep} &= 0.1059.
\end{align}

\begin{center}
\setlength{\tabcolsep}{4pt}
\begin{tabular}{l l r}
\hline
Metric & Visual scale (0 to 0.60) & Value \\
\hline
TPR ($\nu=0.2$) & \rule{4.58cm}{1.8ex} & 0.3432 \\
TPR ($\nu=0.5$) & \rule{5.07cm}{1.8ex} & 0.3805 \\
TPR ($\nu=0.8$) & \rule{6.52cm}{1.8ex} & 0.4892 \\
FPR ($\nu=0.0$) & \rule{3.98cm}{1.8ex} & 0.2983 \\
\hline
\end{tabular}
\par\smallskip
{\small \textbf{Figure 1.} Point-level classification rates from the baseline decision-grade run, showing monotonic increase in TPR across relative-truth points and lower absolute-truth FPR.}
\end{center}

\section{Replicate stability}
Three additional full-core seeds were run sequentially with all settings fixed:
\begin{itemize}
\item \texttt{outputs/growth\_mapping\_power\_map\_decision\_replicates\_seq\_20260210\_045915UTC/seed\_20260212/}
\item \texttt{outputs/growth\_mapping\_power\_map\_decision\_replicates\_seq\_20260210\_045915UTC/seed\_20260213/}
\item \texttt{outputs/growth\_mapping\_power\_map\_decision\_replicates\_seq\_20260210\_045915UTC/seed\_20260214/}
\end{itemize}
The separation values were
\begin{equation}
\Delta_{\rm sep}=\{0.1000,\ 0.1067,\ 0.1089\},
\end{equation}
and across all four decision-grade runs (baseline + 3 replicates),
\begin{equation}
\langle \Delta_{\rm sep} \rangle = 0.1054,\qquad \sigma_{\rm run}=0.0038.
\end{equation}

\begin{center}
\setlength{\tabcolsep}{4pt}
\begin{tabular}{l l r}
\hline
Run & Visual scale (0 to 0.12) & $\Delta_{\rm sep}$ \\
\hline
Baseline & \rule{7.06cm}{1.8ex} & 0.1059 \\
Rep-1 & \rule{6.67cm}{1.8ex} & 0.1000 \\
Rep-2 & \rule{7.11cm}{1.8ex} & 0.1067 \\
Rep-3 & \rule{7.26cm}{1.8ex} & 0.1089 \\
\hline
Mean & \rule{7.03cm}{1.8ex} & 0.1054 \\
\hline
\end{tabular}
\par\smallskip
{\small \textbf{Figure 2.} Run-to-run stability of separation across baseline plus three replicate seeds. The spread is small relative to the mean positive separation.}
\end{center}

\section{Interpretation}
This test establishes that the calibrated classifier produces a stable positive separation under independent seeds and fixed rules. The effect is reproducible but moderate, so this result is best interpreted as calibration-grade evidence of nonzero discriminability in the tested model family, not as a standalone physical proof.

\section*{Data and software availability}
Software archive for this pipeline: \href{https://doi.org/10.5281/zenodo.18582609}{doi:10.5281/zenodo.18582609}.

External data/product DOIs used by the underlying stack include:
\begin{itemize}
\item GWTC-3 products: \href{https://doi.org/10.1103/PhysRevX.13.041039}{doi:10.1103/PhysRevX.13.041039}.
\item GLADE+: \href{https://doi.org/10.1093/mnras/stac1443}{doi:10.1093/mnras/stac1443}.
\item Pantheon+ papers and dataset: \href{https://doi.org/10.3847/1538-4357/ac8b7a}{doi:10.3847/1538-4357/ac8b7a}, \href{https://doi.org/10.3847/1538-4357/ac8e04}{doi:10.3847/1538-4357/ac8e04}, \href{https://doi.org/10.5281/zenodo.16365279}{doi:10.5281/zenodo.16365279}.
\item Cosmic-chronometer compilations: \href{https://doi.org/10.1088/1475-7516/2012/08/006}{doi:10.1088/1475-7516/2012/08/006}, \href{https://doi.org/10.1088/1475-7516/2016/05/014}{doi:10.1088/1475-7516/2016/05/014}.
\item BOSS DR12 BAO+FS: \href{https://doi.org/10.1093/mnras/stx721}{doi:10.1093/mnras/stx721}.
\item eBOSS DR16 LRG BAO+RSD: \href{https://doi.org/10.1093/mnras/staa2455}{doi:10.1093/mnras/staa2455}.
\item DESI 2024 BAO analyses: \href{https://doi.org/10.1088/1475-7516/2025/04/012}{doi:10.1088/1475-7516/2025/04/012}, \href{https://doi.org/10.1088/1475-7516/2025/02/021}{doi:10.1088/1475-7516/2025/02/021}.
\item Planck 2018 lensing: \href{https://doi.org/10.1051/0004-6361/201833886}{doi:10.1051/0004-6361/201833886}.
\end{itemize}

\section*{AI-use statement}
AI systems were used extensively for software prototyping, experiment orchestration, diagnostics, and drafting/editing support in preparing this analysis package.

\begin{thebibliography}{9}
\bibitem{gwtc3}
R.~Abbott \textit{et al.} (LIGO Scientific Collaboration, Virgo Collaboration, and KAGRA Collaboration),
Phys. Rev. X \textbf{13}, 041039 (2023),
\href{https://doi.org/10.1103/PhysRevX.13.041039}{doi:10.1103/PhysRevX.13.041039}.

\bibitem{glade}
G.~D\'alya \textit{et al.},
Mon. Not. R. Astron. Soc. \textbf{514}, 1403 (2022),
\href{https://doi.org/10.1093/mnras/stac1443}{doi:10.1093/mnras/stac1443}.
\end{thebibliography}

\end{document}

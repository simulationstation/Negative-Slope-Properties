\documentclass[aps,prd,reprint,superscriptaddress,nofootinbib,floatfix]{revtex4-2}

\usepackage{amsmath,amssymb}
\usepackage{graphicx}
\usepackage{booktabs}
\usepackage{hyperref}
\usepackage{lineno}
\usepackage{url}

\begin{document}
\title{Testing the Bekenstein-Hawking Area Law at Cosmological Scales: A Data-Driven Reconstruction of the Horizon Entropy Current}

\author{Aiden B. Smith}
\affiliation{Independent Researcher}

\begin{abstract}
The Bekenstein-Hawking area law implies a constant horizon entropy slope, $\mu(A)\propto dS/dA=\mathrm{const}$. We test this assumption directly at cosmological scales using a data-driven reconstruction of the apparent-horizon entropy current from late-time background geometry (Pantheon+, BAO, and cosmic chronometers), in a framework designed to avoid GR-anchored circularity. The reconstruction prefers a time-evolving entropy slope with a negative mean gradient, $\langle d\log\mu/d\log A\rangle\approx -0.24$, with posterior weight $P(d\log\mu/d\log A<0)\approx 0.79$. Function-space proximity metrics favor non-extensive templates (Tsallis/Barrow-like) over strict Bekenstein-Hawking behavior, with representative distances $D^2\sim10^{-8}$ versus $D^2\sim10^{-4}$ for $\mu=1$. Synthetic-closure and simulation-based calibration checks are used to verify that the recovered trend is not a reconstruction artifact. These results support a thermodynamic interpretation in which the effective gravitational sector evolves with horizon area, providing a direct theoretical bridge to modified-propagation (friction) signatures in dark-siren analyses.
\end{abstract}

\maketitle
\linenumbers
\modulolinenumbers[5]

\section{Introduction}
The thermodynamic interpretation of gravity links horizon geometry to entropy. In standard General Relativity with Bekenstein-Hawking entropy, the horizon entropy is proportional to area, $S\propto A$, so $dS/dA$ is constant. A central question for late-time cosmology is whether this area law remains valid on the apparent horizon traced by the observed expansion history.

This paper is framed as a direct test of that assumption. Rather than starting from a fixed gravitational field equation and fitting parameters, we reconstruct the entropy-current proxy $\mu(A)$ from observed background geometry and then ask whether $\mu(A)=1$ is supported. In this sense, the analysis is not only a reconstruction exercise; it is a fundamental consistency test of space-time thermodynamics at cosmological scales.

The same thermodynamic degree of freedom is also the ingredient that controls modified gravitational-wave propagation in dark-siren sectors. The present work therefore serves as the theory-facing companion to dark-siren propagation tests: here we infer the background thermodynamic current, while dark sirens test its propagation consequence.

\section{Thermodynamic Link and Inference Strategy}
We define the entropy-slope modification
\begin{equation}
\mu(A)\equiv \frac{(dS/dA)_{\rm BH}}{dS/dA},
\end{equation}
so that $\mu=1$ corresponds to strict Bekenstein-Hawking behavior. In the apparent-horizon mapping used in this pipeline, the late-time background obeys
\begin{equation}
\frac{dH^2}{dz}=3H_0^2\Omega_{m0}(1+z)^2\mu(A),\qquad
A(z)=4\pi\left(\frac{c}{H(z)}\right)^2
\end{equation}
for the flat mapping variant.

The inference is performed by forward modeling: we sample a spline representation of $\log\mu$ and solve for $H(z)$, then evaluate the likelihood in data space. The baseline geometry stack is Pantheon+, BAO, and cosmic chronometers. Inference products are stress-tested with synthetic closure and simulation-based calibration (SBC), including nominal 68\% and 95\% coverage diagnostics, to ensure that apparent trends are not driven by inversion pathologies.

\section{Results}
\subsection{Area-law test from reconstructed entropy current}
The reconstructed posterior favors a negative entropy-current slope over a constant-slope area law. A representative summary is
\begin{equation}
\left\langle\frac{d\log\mu}{d\log A}\right\rangle \approx -0.24,\qquad
P\!\left(\frac{d\log\mu}{d\log A}<0\right)\approx 0.79.
\end{equation}
Interpreted physically, this corresponds to a decreasing $\mu(A)$ as the horizon area grows, i.e., an evolving entropy density on cosmological horizons.

Function-space proximity tests provide a complementary model-comparison view. Relative to a strict Bekenstein-Hawking template ($\mu=1$), the reconstructed curve is substantially closer to non-extensive families:
\begin{equation}
D^2_{\rm BH}\sim10^{-4},\qquad D^2_{\rm Tsallis/Barrow}\sim10^{-8},
\end{equation}
indicating that non-extensive effective descriptions capture the reconstructed shape more efficiently.

\subsection{Method calibration and identifiability gate}
To verify that sign-level statements are identifiable under finite Monte Carlo noise, we retain the power-map calibration used in the production pipeline. With fixed threshold $\tau=0.06459924567842923$, the baseline run gives
\begin{align}
\mathrm{TPR}_{\nu=0.2}&=0.3432,\quad
\mathrm{TPR}_{\nu=0.5}=0.3805,\quad
\mathrm{TPR}_{\nu=0.8}=0.4892,\\
\mathrm{FPR}_{\nu=0.0}&=0.2983,\quad
\Delta_{\rm sep}=0.1059.
\end{align}
Three additional full-core replicate seeds yield $\Delta_{\rm sep}=\{0.1000,0.1067,0.1089\}$, giving
\begin{equation}
\langle \Delta_{\rm sep}\rangle = 0.1054,\qquad \sigma_{\rm run}=0.0038.
\end{equation}
This establishes stable, nonzero discrimination power for the mapping-choice gate used upstream of decision-grade runs.

\subsection{Hubble-scale anchors and robustness context}
For transfer-map stress tests, we track the conventional anchor scales $H_0^{\rm Planck}=67.4\pm0.5$ and $H_0^{\rm local}=73.0\pm1.0$ km\,s$^{-1}$\,Mpc$^{-1}$ as external references. These anchors are used for sensitivity mapping, not as hard priors in the core area-law test, to preserve modified-gravity neutrality in the reconstruction stage.

\section{Discussion}
The key physical output is not only ``negative slope detected,'' but what that slope means. A decreasing $\mu(A)$ implies that the effective entropy current changes as the Universe expands. In modified-gravity language this maps to an evolving effective Planck sector (running gravitational coupling), a known channel in scalar-tensor and $f(R)$-like effective descriptions.

The non-extensive proximity ordering is consistent with that interpretation. Tsallis/Barrow-like templates are not asserted here as unique microphysical theories; rather, they provide compact effective parameterizations of the reconstructed thermodynamic flow. Their significantly smaller $D^2$ values relative to strict Bekenstein-Hawking behavior quantify that the data-preferred trajectory is not well described by $\mu=\mathrm{const}$.

The calibration and closure layers are included for this reason: to separate physics statements from pipeline artifacts. Synthetic closure and SBC checks, together with seed-level stability of the identifiability gate, support interpreting the negative-slope trend as a property of the reconstructed geometry rather than a numerical accident.

\section{Conclusion}
We recast late-time entropy-slope reconstruction as a fundamental cosmological test of the Bekenstein-Hawking area law. In the reconstructed posterior, the horizon entropy current is not strictly constant and is instead biased toward a negative slope with non-extensive effective behavior favored in function space.

Most importantly, this thermodynamic evolution ($\mu(A)\neq\mathrm{const}$) provides the theory-side basis for the modified gravitational-wave propagation (friction) signals targeted in dark-siren catalogs, linking horizon thermodynamics and multimessenger cosmology in a single framework.

\section*{Data and software availability}
Software archive for this pipeline (Zenodo title: \textit{Negative Entropy Slope}): \href{https://doi.org/10.5281/zenodo.18604922}{doi:10.5281/zenodo.18604922}.

External data/product DOIs used by the underlying stack include:
\begin{itemize}
\item GWTC-3 products: \href{https://doi.org/10.1103/PhysRevX.13.041039}{doi:10.1103/PhysRevX.13.041039}.
\item GLADE+: \href{https://doi.org/10.1093/mnras/stac1443}{doi:10.1093/mnras/stac1443}.
\item Pantheon+ papers and dataset: \href{https://doi.org/10.3847/1538-4357/ac8b7a}{doi:10.3847/1538-4357/ac8b7a}, \href{https://doi.org/10.3847/1538-4357/ac8e04}{doi:10.3847/1538-4357/ac8e04}, \href{https://doi.org/10.5281/zenodo.16365279}{doi:10.5281/zenodo.16365279}.
\item Cosmic-chronometer compilations: \href{https://doi.org/10.1088/1475-7516/2012/08/006}{doi:10.1088/1475-7516/2012/08/006}, \href{https://doi.org/10.1088/1475-7516/2016/05/014}{doi:10.1088/1475-7516/2016/05/014}.
\item BOSS DR12 BAO+FS: \href{https://doi.org/10.1093/mnras/stx721}{doi:10.1093/mnras/stx721}.
\item eBOSS DR16 LRG BAO+RSD: \href{https://doi.org/10.1093/mnras/staa2455}{doi:10.1093/mnras/staa2455}.
\item DESI 2024 BAO analyses: \href{https://doi.org/10.1088/1475-7516/2025/04/012}{doi:10.1088/1475-7516/2025/04/012}, \href{https://doi.org/10.1088/1475-7516/2025/02/021}{doi:10.1088/1475-7516/2025/02/021}.
\item Planck 2018 lensing: \href{https://doi.org/10.1051/0004-6361/201833886}{doi:10.1051/0004-6361/201833886}.
\end{itemize}

\section*{AI-use statement}
AI systems were used for drafting support and editing during manuscript preparation.

\begin{thebibliography}{9}
\bibitem{gwtc3}
R.~Abbott \textit{et al.} (LIGO Scientific Collaboration, Virgo Collaboration, and KAGRA Collaboration),
Phys. Rev. X \textbf{13}, 041039 (2023),
\href{https://doi.org/10.1103/PhysRevX.13.041039}{doi:10.1103/PhysRevX.13.041039}.

\bibitem{glade}
G.~D\'alya \textit{et al.},
Mon. Not. R. Astron. Soc. \textbf{514}, 1403 (2022),
\href{https://doi.org/10.1093/mnras/stac1443}{doi:10.1093/mnras/stac1443}.
\end{thebibliography}

\end{document}
